
\section{Introduction}

% increasing amounts of unstructured text data
The field of database management has traditionally focused on
structured data, providing little or no help for the significantly
larger amounts of the world's data that is unstructured. For many
applications, unstructured text and structured data are both
important natural resources to fuel data analysis. For example, a
sports journalist covering NFL (National Football
League\footnote{http://www.nfl.com} - 32 American football teams with
more than 1700 players) games would need a system that can analyze both the
structured statistics (e.g., scores, biographic data) of teams and players and the
unstructured tweets, blogs, and news about the games.

In such applications, text analysis are performed over text data
from many sources. Text analysis uses the state-of-the-art
statistical machine learning (SML) methods to extract structured
information, such as entities, relations, sentiments, topics, from
text. The result of the text analysis can be joined with other
structured data sources to perform analysis. For example, the sports
journalist may want to join the sentiment result from tweets and the
scores of the Arizona Cardinals\footnote{http://www.azcardinals.com} 
(NFL Team) to see their correlations.

To our knowledge, there is no integrated system with a query
interface that enables domain experts (e.g., journalists) to issues
such ad hoc exploratory queries. To answer such queries, a software
developer is needed to understand and connect multiple tools,
including Lucene for text search, Weka or MATLAB for sentiment
analysis, and a database for joining the structured data with the
sentiment results. Using such a complex offline batch process to 
answer a single query
makes it difficult to ask ad hoc queries over ever-evolving text data. 
These queries are essential for applications, such as computational journalism,
e-discovery, and political campaign management, where queries are
exploratory in nature, and follow-up queries need to be asked based
on the result of previous queries.

\eat{ and an analytic toolkit such as . To support parallelization
for scalability, frameworks such as Hadoop/MapReduce are also
required. \eat{A developer would first create indexes for the
textdata in Lucene to retrieve documents that contain the team
``Arizona Cardinal''. Such retrieval also needs to perform entity
resolution to count for typos and aliases. After the documents are
retrieved a developer can use an off-the-shelf model and algorithm
to calculate the sentiment.} Finally, the sentiment results are
imported into a database to be joined with the score statistics.}

% need of query-driven analysis
% higher-level of abstraction of ML models using simple ER models and SQL queries
% real-time analysis enable exploratory queries and query refinement based on result
% many applications: computational journalism, campaign management, e-discovery, etc.

In this demonstration paper, we describe \system, an integrated
system that supports both relational query processing and
statistical analysis over text. \system is implemented on top of MADLib,
which is collection of libraries developed on 
top of PostgreSQL/Greenplum. MADLib supports
in-database statistical methods, and in particular, the work on in-database text extraction
 from BayesStore~\cite{Wang:2008:BML:1453856.1453896} is being integrated into
 MADLib. \system adapt the statistical methods in MADLib for various text analysis tasks, including classification using Naive Bayes, information extraction using CRF,
and add entity resolution functionality using q-grams.

\eat{
% queries supported by madden

% main techniques to support madden

Relational views are created representing the results of statistical
text analysis over textual data involving different models.

To support this over large data we need parallel algos. }

\system supports a declarative query interface and real-time query
processing, which is needed for ad hoc analysis. SQL-like queries
and ER (Entity-Relationship) models are used as a higher-level of
abstraction of the SML models and inference methods for text
analysis. ad hoc query-driven text analysis enable exploratory
queries and query refinement based on the results. Users of such
query interfaces do not have to know the underlying statistical
machine learning models or the various text analysis tasks being
exercised by the systems. This interface provide a clean abstraction
from the detailed text analysis algorithms from the
application-specific data exploration.

\eat{ Using {\system} we are able to perform a declarative query
over the data set to return the set of correct documents. Using
native SQL statements, we get a query optimizer that allows us to
perform each step of the query with out iteratively waiting to load
all documents. Another benefit of {\system} is we bring the query to
the data, there is no needed to make large copies spend time
transferring portions of the data around. For large data sets, the
data becomes unwieldy for in memory implementations. Using
{\system}, the system and any algorithms scale with the data. A user
does not have to rely on samples, as commonly done using analytic
tools. We provide a framework for data analyst to ingest and analyze
the information on the fly. The system that hold the data is the
same system that is queries. To support adhoc textual analysis over
large data sets a query-driven approach is necessary for real-time
exploration. During the computation we can use the information given
in the query to reduce total computation.}

% main approach and techniques of MADden
\eat{In this paper we describe \system, a system that allows
advanced querying over structured and unstructured data. Beyond
standard SPJ (Select-Project-Join) queries we demonstrate inference
algorithms using native database techniques. Additionally, we
implementation is over PostgreSQL and Greenplum parallel databases,
demonstrating the scalability of our system.}

In the demonstration of \system, we will show the 
following points in two application domains, including the NFL
and the presidential election.
\begin{itemize}
\item Real-time processing of ad hoc queries 
involving statistical text analysis, such as sentiment analysis,
information extraction, and entity resolution;
\item Joining and aggregation between structured and information extracted
from textual data;
\item Query-time rendering of visualizations over query results, using 
word clouds, histograms, and ranked lists of documents.
\end{itemize}

