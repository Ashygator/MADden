
\section{System Description}

In this section we first discuss the general architecture the {\system} 
system. Next, we discuss the spirit and motivation behind the \system 
implementation of the text analysis algorithms.
Next, we discuss how \system excels with the difference classes of algorithms
implemented. 
Then we discuss how parallel database architectures improve query 
processing times.
Lastly, we propose our part-of-speech (POS) tagging implementation to 
give a practical example of a \system algorithm implementation.


\begin{figure}
   \begin{center}
        \includegraphics[scale=0.2]{content/graphics/arch.png}
        \caption{{\system} architecture}
        \label{fig:arch}
   \end{center}
\end{figure}
    
\subsection{System Architecture}

{\system} is a four layered system, as can be seen in Figure \ref{fig:arch}.
The user interface is where both naive and advanced users can construct queries over
text, structured data, and models. From the user interface, queries are then passed to the
DBMS, where both MADLib and {\system} libraries sit on top of the query processor to
add statistical and text processing functionality. 
It is important to emphasize that MADlib and {\system} perform functions
at the same logical layer.  These queries are processed using 
PostgreSQL and Greenplum's Parallel DB architecture to further optimize on 
replicated storage and parallel query optimization. 


%\subsection{{\system} Implementation}



% -------------------------------------------- Rewrite




% All new stuff-------------% -------------% ----------------
\subsection{In-datbase Implementation}
To enable text analytics, {\system} works alongside statistical
functions found in the MADlib library \cite{Cohen:2009:MSN:1687553.1687576}.
The implemented algorithms are developed to take advantage of the 
architecture and features of the DBMS.
The algorithms are first implemented in SQL or PL/pgSQL, as this language
exposes the execution pipeline to the query optimizer. 
SQL excels in first-order logic operations performed in batch.
If many loops and complicated data flows are involved, we use the PL/Python 
scripting language to control the execution.
We can use Python to make decisions to aid the query optimizer in algorithm
decision.
Dense or sparse array and matrix data structures can be represented
by specialized types. Operations on these structures can be performed in
SQL or calculated with calls to highly tuned numeric packages such as LAPACK
or custom c-UDFs.
During development of algorithms in MADden or MADLib we take these options 
into consideration.

Table \ref{tab:madschema} describes a list of statistical text 
analysis tasks.
{\system}.


\subsection{Query-driven Statistical Text Analysis}

In this section we describe the {\system} algorithms.
Many approaches exist for in-database information extraction. 
We build on our previous work using Conditional Random Fields (CRFs) for 
query-time information extraction \cite{wang2011hybrid}.
We use an hybrid approach to information extraction where we perform the 
extraction and the inference inside of the database. We rely on information
provided in the query to make decision on the type of algorithm used 
for extraction. 
We discuss this in detail in section \ref{sec:impdis}.

In {\system} we encounter situations requiring entity resolution problem.
Entity resolution or co-reference resolution is the problem where given any two
mentions of a name, the are clustered only if they refer to the same real 
world entity. This problem is difficult because you can have mentions that are
spelled the same but refer to different entities and conversely we different
spelling that may refer to the same entity.
Certain entities may be misrepresented by the presence of different names, 
misspellings in the text, or aliases. It is important to
resolve these entities appropriately to better understand the data. Increasingly
informal text, such as blog posts and tweets requires entity resolution. 
{\system} utilizes inverted indices within the database to perform text
analysis on documents. We can scan the inverted index of each 
document, filtering out documents that do not contain instances of the player names.
To handle misspellings and nicknames we use trigram indices to perform 
approximate matches of searches for names \cite{Jain:2009:BQO:1519103.1519108}.
This method allows us to use indices to perform queries on only the relevant
portions of the data set, this way we do no extra work.

Each text analysis function creates a view that can continue to be called.
Since views can be queried in the same manner as tables, we can join calculated
statistics with other tables, or other text analysis views, to construct more
interesting results. This, combined with expressing text analysis functionality
within a SQL query, allows for more complex queries to be realized. 



\subsection{Parallelization}
With the ability to implement a parallel database architecture with %Postgres or 
Greenplum, we can utilize data parallelism to further optimize queries written
with {\system}. Each node within the parallel DB could run some query
over a subset of the data (data parallel). This includes the statistical methods
in MADLib, which were all built to be data parallel.

In POS tagging of a corpra is an integral task for many 
textual analytic operations. POS tagging can be performed per-paragraph or 
per-sentence meaning the process is easily parallelizeable in our queries.

With POS tagging performed we can perform the similarly parallel operation 
of sentiment analysis.
Sentiment analysis deals with the study of opinions expressed within text. 
The sentiment is usually expressed in terms of polarities, where positive
sentiment refers to the positive nature of the expressed opinion, and negative
nature for negative sentiment. Much work has already been done in this
area for document-level and entity-level sentiment \cite{o2010tweets,
zhang2011combining}. 
{\system} can apply sentiment analysis at document and sentence level.


\subsection{Implementation Details}
\label{sec:impdis}

%%%---------new pos section

Core to many natural language processing tasks, POS involves the
labeling of terms within text based on their function in a particular sentence.
We implemented POS tagging in Postgres and Greenplum. We are
committing our code to MADLib and it is in under review.

\system uses first order chain conditional random field to model the labeling
of a sequence of tokens. The factor graph has observed nodes on each sentence
token, latent label variables attached to each token.
Factors are only associated with one node at the beginning and end of the chain.
All other factors are pairwise.  
We generate the features using a function \textit{generatemrtbl}. 
This function produces a table \textit{rfactor} for single state features and a 
table \textit{mfactor} for two state features.

Training the CRF model is a one time task that is performed outside the 
DBMS \footnote{We use the IIT Bombay package for training available at 
http://crf.sourceforge.net }.
We use a small python script to parse and import the trained model into 
tables in the DBMS. 

Inference is performed over the stored models in order to find the highest 
assignment of labels in the model.
We calculate the top 1 most probable label assignment. This is calculated using
the Viterbi dynamic programming algorithm over the label space.


We use the PL/Python language to execute manage the work flow of all the 
calculations.
The computationally expensive function viterbi is implemented as c-UDFs.
The feature generation and execution of inference 
over a table of sentences is implemented in SQL. When executed in GreenPlum 
the query is performed in parallel.


Implementing POS tagging inside the DBMS allows us to perform inference over a 
subset of tokens response to a query instead of performing batch tagging over 
all tokens.
We also get the benefit of using the query engine to parallelize our queries
without losing the ability to drive the work flow using PL/Python.


Example \textit{Q0} performs POS tagging for  all the sentences that contain
the word 'Jaguar'. This query interface allows the user to perform
functions on a subset of the data.
The \textit{segmenttbl} holds a list of words and their position for each
document (\textit{doc\_id}). We assume a document is an sequence of characters.



\begin{small}
\begin{alltt}
\textit{Q0: POS tagging on sentences with the word `Jaguars'}
SELECT DISTINCT ON segtbl.doc_id,
    viterbi(segtbl.seglist,mfactor.score,rfactor.score)
FROM segmenttbl, mfactor, rfactor, 
    (SELECT doc_id, array_agg(seg_id ORDER BY start_pos)
			AS seglist
     FROM segtbl 
     GROUP BY doc_id) AS segtbl
WHERE segtbl.doc_id = segmenttbl.doc_id 
    AND segmenttbl.seg_text='Jaguar';
\end{alltt}
\end{small}

%-----------end new pos section











