
\section{Text Analysis Queries}

In this section, we exemplify the type of text analysis queries
supported by the MADden system using a running example from the
application domain of computational journalism.

Suppose a sports journalist needs to do some research on public opinion of the
Arizona Cardinals for the current season, based on recent news, blogs and
tweets.

\eat{ NFL domain is centered around 32 \textit{teams} and more than 1700
\textit{players}. Each team in the league plays 4 preseason
\textit{games} and 16 regular season games. The top teams each play a
\textit{tournament} which concludes in a final game known as a ``Super
Bowl''. Additionally, each team has one \textit{head coach} and
several \textit{assistant coaches}. }


\eat{ In addition to the various entities in the domain (e.g.,
teams, players), the players and teams also have aliases. For
example, player ``X'' is also called ``Y'', and team ``A'' is also
called ``B''. In text analysis, we not only need to extracted the
different entities in this domain, we also need to resolute the
different names of the same entity.}

\begin{table}
\begin{center}
\begin{tabular}{|l|}
\hline
\multicolumn{1}{|c|}{Schema}\\
\hline
NFLCorpus (id, type, entry, title, posted\_at)\\
\hline
Players (id, name)\\
\hline
PlayerStats (pid, position, stats\ldots)\\
\hline
PlayerAlias (pid, nickname)\\
\hline
Teams (id, name, city)\\
\hline
TeamAlias (id, nickname)\\
\hline
GameStats (id, game\_date, hid, aid, hscore, ascore)\\
\hline
\end{tabular}
\end{center}
\caption{The abbreviated example schema}
\label{tab:schema}
\end{table}

The data is represented by the abbreviated schema\footnote{These tables may be
extracted to an RDBMS beforehand, or defined over an API
using a foreign data wrapper (fdw).} shown in Table \ref{tab:schema}.
The {\tt NFLCorpus} table holds documents crawled from the web,
and tweets extracted from the Twitter Streaming API. This table contains
millions of documents, varying in size and quality. The type of document 
can be specified via the type attribute. The types include `news', `blogs' or
`tweets'. The \textit{entry} attribute of {\tt NFLCorpus} contains the text data.
The other tables are structured but the data may contain duplicates or
misspelled terms. It is also possible for these documents to be out of domain.
The {\tt Players} table is a list of individuals
who play in the NFL. The player data is obtained from NFL.com
The {\tt PlayerStats} table has a \textit{position} field that specifies
the type of statistic that is inside of the \textit{stats} column.
The {\tt PlayerAlias} table list alternative names for players.
While this list may be obtained using topic modeling or word
co-occurrence, we hand coded a list of alias names for a select
group of well-known players. This helps in resolving players who are
referred to by their nickname, `Sticky Fingers', in the case of
`Larry Fitzgerald'.
The {\tt Teams} table holds all the team
names in the NFL and the {\tt TeamAlias} table has the alternate names for these
teams. ``ARI'' as an abbreviation,  and ``Football Cardinals'' as a nickname, are 
example entries in this table. {\tt GameStats} contains the amount of points scored by a 
home and away team for each NFL game played.\\


In addition to the table, we have several UDFs that perform several statistical
text analysis tasks. The {\tt match} function takes two strings and
uses alias tables to perform a fuzzy match.
Additionally, {\tt extract\_entities} is a materialized view that is the
result of the information extraction process in the vein of previous work
\cite{wang2010probabilistic}.
It returns tuples with the schema
(\textit{docid}, \textit{pos}, \textit{token}, \textit{entity}) where the
columns respectively represent
the document id (e.g. article url), the position in a document, the token
in said document and the entity id to which it corresponds.
The {\tt sentiment} UDF is a sentiment analysis classifier.
It takes a string or entity as the first parameter and as the second parameter
it takes the textual entry. It returns \textit{positive} if the sentiment is
positive and \textit{negative} otherwise.

%In this example query we demonstrate the flexibility and multi-modal
%support of \system.
Our running query is an example of a question that may be posed by
the journalist ``{\tt Give me the positive news involving the Arizona Cardinals
NFL team and involving the player Larry Fitzgerald.}''

\eat{
\dzw{query templates should be an interface issue, to be explained
in the demo section. we can say we support both query interface as
well as web interface with query templates for executive users} Each
parameter to the queries is demarcated by underlines. The qualifier
{\tt best} for news is interpreted as news articles that have the
highest ratio of good news to bad news.}

In this domain, an example of positive news is a team victory, a player
making an important play such as a last second score, or contributions to the
communities they represent (i.e. volunteering, charities).
{\tt Arizona Cardinals} is a team in the NFL, any team could be selected
in this parameter.
The parameter {\tt player} is associated with
{\tt Larry Fitzgerald} who plays for the Cardinals. The player can be any
arbitrary player. In place of a player, the user can select a {\tt coach}
or an arbitrary {\tt person}. The names of players or coaches may also be
misspelled or simply not unique, so an entity resolution step over
the documents will be necessary.



%\lstset{breaklines=true}
%\lstset{tabsize=2}
%\lstset{basicstyle=\small}
%\begin{lstlisting}{language=SQL}
\begin{small}
\begin{alltt}
\textit{Q1: [Ranked Document Lists]}
SELECT e.freq, c.docid,  e.entry
FROM NFLCorpus c
    (SELECT docid, count(*) AS freq
    FROM extract_entities
    WHERE match('LarryFitzgerald',entity)) AS e
WHERE sentiment(entry) = '+',
    AND match('Arizona Cardinals', entity) is true
    AND type = 'news' AND c.docid = e.docid
ORDER BY e.freq
\end{alltt}
\end{small}
%\end{lstlisting}

Other queries that can be answered by \system include aggregates and joins
between structured and text based data.

This query counts the number of times each entity is mentioned in either tweets
or blogs. This query performs an aggregate over a fuzzy entity resolution
task across multiple media. The result of this query could be the input to a
word cloud or a histogram for fast visual analysis.
%\lstset{tabsize=2}
%\lstset{breaklines=true}
%\lstset{basicstyle=\small}
%\begin{lstlisting}{language=SQL}
\begin{small}
\begin{alltt}
\textit{Q2: [Aggregation Query]}
SELECT entity, SUM(count(t.docid) + count(b.docid))
FROM NFLCorpus c, extract_entities t, extract_entities b
WHERE c.type = 'tweets' OR c.type = 'blogs' 
  AND t.entity = b.entity AND match(ct.entity, t.entity)
GROUP BY entity
HAVING freq > 10
ORDER BY posted_at DESC;
\end{alltt}
\end{small}
%\end{lstlisting}

In this query we do a join over structured data and text to return the
sentiment towards a team vs. their performance at home. 
%\lstset{tabsize=2}
%\lstset{breaklines=true}
%\lstset{basicstyle=\small}
%\begin{lstlisting}
\begin{small}
\begin{alltt}
\textit{Q3: [Structured and Unstructured Data]}
SELECT ts.pid, ts.date, g.hscore / g.ascore AS perf,
  SUM(sentiment(c.entry))
FROM Team t, TeamStats ts, NFLCorpus c, GameStats g
WHERE c.type = 'tweet' AND t.name = 'Cardinals' 
  AND t.id = g.hid AND gs.game_date = c.posted_at 
  AND match(t.name,c.entry) IS true
GROUP BY ts.pid, ts.date, g.hscore, g.ascore
ORDER BY gs.game_date
\end{alltt}
\end{small}
%\end{lstlisting}


This query is typical of a sports agent or a public relations representative
monitoring the overall public opinion of a team. A query interface provides a
simple method for real-time monitoring. 
Multi-modal queries, like this, allows us to find relationships and calculate
these statistics within one system. 



