\section{Datasets}
The data is represented by the abbreviated schema\footnote{These tables may be
extracted to an RDBMS beforehand, or defined over an API
using a foreign data wrapper (fdw).} shown in Table \ref{tab:schema}.
The {\tt NFLCorpus} table holds documents crawled from the web,
and tweets extracted from the Twitter Streaming API. This table contains
millions of documents, varying in size and quality. The type of document 
can be specified via the type attribute. The types include `news', `blogs' or
`tweets'. The \textit{entry} attribute of {\tt NFLCorpus} contains the text data.
The other tables are structured but the data may contain duplicates or
misspelled terms. It is also possible for these documents to be out of domain.
The {\tt Players} table is a list of individuals
who play in the NFL. The player data is obtained from NFL.com
The {\tt PlayerStats} table has a \textit{position} field that specifies
the type of statistic that is inside of the \textit{stats} column.
The {\tt PlayerAlias} table list alternative names for players.
While this list may be obtained using topic modeling or word
co-occurrence, we hand coded a list of alias names for a select
group of well-known players. This helps in resolving players who are
referred to by their nickname, `Sticky Fingers', in the case of
`Larry Fitzgerald'.
The {\tt Teams} table holds all the team
names in the NFL and the {\tt TeamAlias} table has the alternate names for these
teams. ``ARI'' as an abbreviation,  and ``Football Cardinals'' as a nickname, are 
example entries in this table. {\tt GameStats} contains the amount of points scored by a 
home and away team for each NFL game played.\\