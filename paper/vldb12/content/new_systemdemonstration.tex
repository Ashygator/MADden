\section{Text Analysis Queries and Demonstration}
Suppose a sports journalist wants to do an investigative
piece on overall public opinion of the Jacksonville Jaguars
for the 2011-2012 season.
\jag{Do we still need to use az as an example now that we are submitting to
vldb??}
Such a peice requires an in-depth analysis of news reports, fan tweets, and 
football (NFL) blogs. And suppose we would also like to correlate the team's
performance to the public opinion. Such a query currently would require
the combination of multiple tools, and \textit{code glue} to solve the problem.
However with the MADden system, these queries require much less effort with SQL.
Consider the simple query \textbf{Give me the positive sentiment involving the
Jacksonville Jaguars and 'Maurice Jones-Drew.'} With MADden, a query like
this be expressed as in Query 1. Such a simple query shows the power of a 
system like MADden. As with any SQL query, we have the ability to build much 
more complex queries through the use of smaller, simpler queries.

%\lstset{breaklines=true}
%\lstset{tabsize=2}
%\lstset{basicstyle=\small}
%\begin{lstlisting}{language=SQL}
\begin{small}
\begin{alltt}
\textit{Q1: [Ranked Document Lists]}
SELECT e.freq, c.docid,  e.entry
FROM NFLCorpus c
    (SELECT docid, count(*) AS freq
    FROM extract_entities
    WHERE match('Jacksonville Jaguars', entity)) AS e
WHERE sentiment(entry) = '+',
AND match('Maurice Jones-Drew', entity)
ORDER BY e.freq
\end{alltt}
\end{small}
\label{}
%\end{lstlisting}
\jag{Why exactly does the match function have an entity flag attached to it??}
\jag{Maybe a contains method as well }