\section{Text Analysis Queries and Demonstration}
Suppose a sports journalist wants to do an investigative
piece on overall public opinion of the Jacksonville Jaguars
for the 2011-2012 season.
\jag{Do we still need to use az as an example now that we are submitting to
vldb??}
Such a peice requires an in-depth analysis of news reports, fan tweets, and 
football (NFL) blogs. And suppose we would also like to correlate the team's
performance to the public opinion. Such a query currently would require
the combination of multiple tools, and \textit{code glue} to solve the problem.
However with the MADden system, these queries require much less effort with SQL.
Consider the simple query \textbf{Give me the positive sentiment involving the
Jacksonville Jaguars and 'Maurice Jones-Drew} (Query 1).' 

%\lstset{breaklines=true}
%\lstset{tabsize=2}
%\lstset{basicstyle=\small}
%\begin{lstlisting}{language=SQL}
\begin{small}
\begin{alltt}
\textit{Q1: [Ranked Document Lists]}
SELECT e.freq, c.docid,  e.entry
FROM NFLCorpus c
    (SELECT docid, count(*) AS freq
    FROM extract_entities
    WHERE match('Jacksonville Jaguars', entity)) AS e
WHERE sentiment(entry) = '+',
AND match('Maurice Jones-Drew', entity)
ORDER BY e.freq
\end{alltt}
\end{small}
\label{}
%\end{lstlisting}
\jag{Why exactly does the match function have an entity flag attached to it??}
\jag{Maybe a contains method as well }

MADden's text analytics package is built alongside MADlib, through the use of
user defined functions (UDFs). As with the existing MADlib toolkit, MADden's 
declarative nature, allows the user to push queries through their datasets,
rather than attempting to fit the data to a particular function or package.
This becomes increasingly advantageous, as in many fields, such as sports
journalism, investigative pieces are a combination of textual analysis and
statistical insight. Combining both tasks is done in the same fashion as
using subqueries in SQL.

