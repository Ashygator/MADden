\section{Text Analysis Queries and Demonstration}
[Key points to think about]
[How to present this as easy to use?]
[How to carry the sports journalism story throughout]

Based on our example dataset, suppose a sports journalist
wants to do an investigative piece on overall public opinion 
of all the Florida NFL teams during the 2011-2012 season. 
Such a peice would require in-depth analysis of news reports,
tweets, blog postings, among other sources. On the text
analytics side of this problem, the standard approach would
consist of trawling through the text sources either by hand
with help, or utilizing a series of different text processing
toolkits and packages, sometimes specialized for a singular 
task. MADden can streamline this process, with its declarative,
in-database approach to text analytics. A first step could 
conceivably consist of paring your corpora down to just
the documents ``related'' to the Florida football teams, namely
the Miami Dolphins, Jacksonville Jaguars, and Tampa Bay Buccaneers. 

%\lstset{breaklines=true}
%\lstset{tabsize=2}
%\lstset{basicstyle=\small}
%\begin{lstlisting}{language=SQL}
\begin{small}
\begin{alltt}
\textit{Q1: Entity Resolution}
SELECT DISTINCT docid
FROM extracted_entities
WHERE match('Jacksonville Jaguars', entity)
   OR match('Miami Dolphins', entity)
   OR match('Tampa Bay Buccaneers', entity);
\end{alltt}
\end{small}
%\end{lstlisting}

We note that match(TARGET_ENTITY, current_entity), is an Entity
Resolution UDF, and ExtractedEntities is a view constructed 
using an Entity Recognition function on textual documents as they
are added to the database (In this case, likely news articles and blogs). A
table of all current text analysis functions can be seen in Table 2.\\

[this is where we want to put the table\ldots]

Maybe the journalist wants to explore fan sentiment for the Jacksonville Jaguars
based on tweets collected during the NFL season. Utilizing the first 
query as a building block (with some changes), we can construct this 
query as follows:

%\lstset{breaklines=true}
%\lstset{tabsize=2}
%\lstset{basicstyle=\small}
%\begin{lstlisting}{language=SQL}
\begin{small}
\begin{alltt}
\textit{Q2: Entity Resolution and Sentiment Analysis}
SELECT DISTINCT E.docid, E.entity
FROM extracted_entities as E
WHERE match('Jacksonville Jaguars', entity)
   OR match('Jaguars', entity);
   or match('Jags', entity)
INNER JOIN
SELECT S.docid, S.sentiment(document)
FROM tweets as S
HAVING sentiment(document) in ('+', '-')
ON E.docid = S.docid;

\end{alltt}
\end{small}
%\end{lstlisting}

In this case, one could accomodate for nicknames through OR-matching
the extracted entities, an alias table, among other strategies. Notice
that going from a singular text analytics task, to a somewhat more 
complex analysis only required a small change in our SQL query. 
Whereas a traditional approach would have us either looking for a
customized solution or patching together packages, the declarative
sql approach allows the user to just state what the result is


***** CONTINUE HERE *****
[[match]
 [detect]
 [pos_tag]
 [sentiment]
 Maybe we can have more as time goes on\ldots]


Suppose a sports journalist wants to do an investigative
piece on overall public opinion of the Jacksonville Jaguars
for the 2011-2012 season. Such a peice requires an in-depth analysis of news
reports, fan tweets, and football (NFL) blogs. And suppose we would also like 
to correlate the team's performance to the public opinion. Such a query
currently would require the combination of multiple tools, and \textit{code glue}
to solve the problem. However with the MADden system, these queries require much
less effort with SQL. Consider the simple query \textbf{Give me the positive
sentiment involving the Jacksonville Jaguars and 'Maurice Jones-Drew} (Query 1).' 

%\lstset{breaklines=true}
%\lstset{tabsize=2}
%\lstset{basicstyle=\small}
%\begin{lstlisting}{language=SQL}
\begin{small}
\begin{alltt}
\textit{Q1: Sentiment and ER example }
SELECT e.freq, c.docid,  e.entry
FROM NFLCorpus c
    (SELECT docid, count(*) AS freq
    FROM extract_entities
    WHERE match('Jacksonville Jaguars', entity)) AS e
WHERE sentiment(entry) = '+',
AND match('Maurice Jones-Drew', entity)
ORDER BY e.freq
\end{alltt}
\end{small}
\label{}
%\end{lstlisting}
\jag{Why exactly does the match function have an entity flag attached to it??}
\jag{Maybe a contains method as well . and this query just needs to be cleaned
up}

In this query, we have entity resolution (match function), and sentiment
analysis (sentiment function), acting on a series of data tables NFLCorpus and
ExtractEntities \jag{(name needs to change)}

MADden's text analytics package is built alongside MADlib, through the use of
user defined functions (UDFs). As with the existing MADlib toolkit, MADden's 
declarative nature, allows the user to push queries through their datasets,
rather than attempting to fit the data to a particular function or package.
This becomes increasingly advantageous, as in many fields, such as sports
journalism, investigative pieces are a combination of textual analysis and
statistical insight. Combining both tasks is done in the same fashion as
using subqueries in SQL. Now supposing that our sports journalist wants
to obtain positive articles from the top 5 rushers of the 2012 NFL season,
they can do so as follows:

\jag{PLACE THIS QUERY HERE. . .}

%\lstset{tabsize=2}
%\lstset{breaklines=true}
%\lstset{basicstyle=\small}
%\begin{lstlisting}
\begin{small}
\begin{alltt}
\textit{Q3: [Structured and Unstructured Data]}
SELECT ts.pid, ts.date, g.hscore / g.ascore AS perf,
  SUM(sentiment(c.entry))
FROM Team t, TeamStats ts, NFLCorpus c, GameStats g
WHERE c.type = 'tweet' AND t.name = 'Cardinals' 
  AND t.id = g.hid AND gs.game_date = c.posted_at 
  AND match(t.name,c.entry) IS true
GROUP BY ts.pid, ts.date, g.hscore, g.ascore
ORDER BY gs.game_date
\end{alltt}
\end{small}
%\end{lstlisting}

\jag{THis query is not representative of the one discussed\ldots just
placeholder.}

During the conference, we plan to give an interactive demonstration of the 
{\system}'s capabilities.
We built a user interface to allow the user to interact with \system. 
We will allow the user to use tables from our NFL data set or political data
for campaign management queries.

Our demonstration will illustrate the following points: 
\begin{enumerate}
  \item The ability to perform statistical text analytics inside the DBMS
  \item Query driven computation of analytics
  \item The flexibility of our data sources by using structured data and text
  \item Query execution over large amounts of text in real-time
  \item Ability to perform textual analytics across multiple domains
\end{enumerate} 

For the demonstration, a web interface will be provided to interact with our
system. It will utilize both a raw SQL UI, as well as a Mad
Lib\footnote{http://en.wikipedia.org/wiki/Mad\_Libs} style interface. The raw
SQL interface will allow users to type in their own SQL queries to interact with
the datasets, while the Mad Libs interface will allow users to customize query
templates through drop-down lists and text box completions.

Regardless of the interaction type, results from the queries will be presented 
to the user through a vareity of visualizations

